\documentclass{article}
\usepackage[top=1cm, left=1.5cm, right=1.5cm, bottom=1.5cm]{geometry}
\usepackage{graphicx, amsmath, tikz-cd, apacite, amssymb, tcolorbox, wrapfig}
\graphicspath{{./img/}}
\bibliographystyle{apacite}
\setlength{\parindent}{0pt}
\setlength{\parskip}{1em} 
 
\usepackage{listings}
\usepackage{xcolor}

% Configuración para el código SQL
\lstdefinestyle{SQL}{
  language=SQL,
  basicstyle=\ttfamily\small,
  commentstyle=\color{brown},
  keywordstyle=\color{purple},
  numberstyle=\tiny\color{gray},
  numbers=left,
  frame=single,
  breaklines=true,
  breakatwhitespace=true,
  tabsize=2,
}

% Nuevo entorno para el código SQL
\lstnewenvironment{sqlcode}
{\lstset{style=SQL}}
{}


\title{SQL Statements}
\author{Pablo Dario}
\date{05/01/2024}
 
\begin{document}
\maketitle
\section{Types of SQL Statements}
 
SQL statements are used to interacting with entities; there are 2 types of SQL Statements, which are \textbf{Data Definition Language} statements and \textbf{Data Manipulation Language} statements.

\textbf{Data Definition Language} statements or \textbf{DDL} statements are used to define, change or drop database objects, such as tables, some common \textbf{DDL} are:
\begin{itemize}
    \item[-] CREATE
    \item[-] ALTER
    \item[-] TRUNCATE
    \item[-] DROP
\end{itemize}

In the other hand, \textbf{Data Manipulation Language} statements or \textbf{DML} statements are used to read and modify data in tables, these are also referred to as \textbf{CRUD} operations (Create, Read, Update and Delete), common \textbf{DML} statements types include:
\begin{itemize}
    \item[-] INSERT
    \item[-] SELECT
    \item[-] UPDATE
    \item[-] DELETE
\end{itemize}

\section{Data Definition Language}

\subsection{CREATE TABLE}

The \textbf{CREATE TABLE} statement is used for creating entities (tables) in a relational database, which includes definition of attributes (columns) such as:

\begin{itemize}
    \item[-] Name of columns
    \item[-] Datatypes of columns
    \item[-] Constraints (e.g. Primary Key) 
\end{itemize}

\begin{sqlcode}
    CREATE TABLE table_name
    {
        <column_name_1> datatype optional_parameters,
        <column_name_2> datatype, 
        ...
        <column_name_n> datatype
    }
\end{sqlcode}

\begin{large}
    \textbf{Example}
\end{large}

\begin{sqlcode}
    CREATE TABLE author
    (
        author_id CHAR(2) PRIMARY KEY NOT NULL,
        lastname VARCHAR(15) NOT NULL,
        firstname VARCHAR(15) NOT NULL,
        email VARCHAR(15),
        city VARCHAR (15),
        earnings DECIMAL(10,2) DEFAULT 0,
        date of birth DATE,
    );
\end{sqlcode}

\begin{large}
    \textbf{Example PRIMARY KEY consisting of 2 columns}
\end{large}

\begin{sqlcode}
    CREATE TABLE Persons (
    ID int NOT NULL,
    LastName varchar(255) NOT NULL,
    FirstName varchar(255),
    Age int,
    CONSTRAINT PK_Person PRIMARY KEY (ID,LastName)
    );
\end{sqlcode}

\begin{large}
    \textbf{Example FOREIGN KEY}
\end{large}

\begin{sqlcode}
    CREATE TABLE Orders (
    OrderID int NOT NULL,
    OrderNumber int NOT NULL,
    PersonID int,
    PRIMARY KEY (OrderID),
    CONSTRAINT FK_PersonOrder FOREIGN KEY (PersonID)
    REFERENCES Persons(PersonID) ON DELETE SET NULL ON UPDATE CASCADE
    );
\end{sqlcode}

Constraints can be specified when the table is created with the \textbf{CREATE TABLE} statement, or after the table is created with the \textbf{ALTER TABLE} statement.

\subsection{ALTER TABLE}

ALTER TABLE statements used to add or remove columns from a table, to modify the data type of columns, to add or remove keys, and to add or remove constraints.

\begin{large}
    \textbf{ADD Columns}
\end{large}

\begin{sqlcode}
    ALTER TABLE <table_name>
    ADD COLUMN <column_name_1> datatype column_constraints,
    ... 
    ADD COLUMN <column_name_n> datatype column_constraints;
\end{sqlcode}

\begin{sqlcode}
    ALTER TABLE Customers
    ADD Email varchar(255);
\end{sqlcode}

\begin{large}
    \textbf{DROP COLUMN}
\end{large}

\begin{sqlcode}
    ALTER TABLE <table_name>
    DROP COLUMN <column_name>;
\end{sqlcode}

\begin{sqlcode}
    ALTER TABLE Customers
    DROP COLUMN Email;
\end{sqlcode}

\begin{large}
    \textbf{MODIFY COLUMN}
\end{large}

\begin{sqlcode}
    ALTER TABLE table_name
    ALTER COLUMN column_name datatype;
\end{sqlcode}

\begin{sqlcode}
    ALTER TABLE Customers
    ALTER COLUMN Email varchar(50);
\end{sqlcode}

\begin{sqlcode}
    ALTER TABLE Customers
    ALTER COLUMN Country ADD DEFAULT USA; 
\end{sqlcode}

\subsection{DROP and TRUNCATE TABLE}

\textbf{DROP and TRUNCATE TABLE} statements are used to delete all of the rows in a table. The drop statement will delete the entire table, while truncate will delete only the rows, but not the table itself. 

\begin{sqlcode}
    DROP TABLE <table_name>;
\end{sqlcode}

\begin{sqlcode}
    TRUNCATE TABLE <table_name>;
\end{sqlcode}

\end{document}