\documentclass{article}
\usepackage[top=1cm, left=1.5cm, right=1.5cm, bottom=1.5cm]{geometry}
\usepackage{graphicx}
\usepackage{wrapfig}
\graphicspath{{./img/}}
\usepackage{amsmath}
\usepackage{tcolorbox}
\setlength{\parindent}{0pt}
\setlength{\parskip}{1em} 
 
\title{Sistemas Homogéneos de Ecuaciones}
\author{Pablo DS}
\date{06/12/2023}
 
\begin{document}
\maketitle

Un sistema general de $m \times n$ ecuaciones lineales se llama homogéneo si todas las constantes $b_1, b_2,\dots, b_m$ son cero; si alguna o algunas de las constantes  $b_1, b_2,\dots, b_m$ es o son diferentes de cero, decimos que el sistema lineal es no homogéneo. Es decir el sistema general homogéneo está dado por: 

\begin{equation*}
    \begin{matrix}
        \begin{aligned}
            a_{11}x_1 + a_{12}x_2 + \dots + a_{1n}x_n = 0\\
            a_{21}x_1 + a_{22}x_2 + \dots + a_{2n}x_n = 0\\
            \vdots \phantom{aaaaaaaa} \vdots \phantom{aaaaaaaaaa} \vdots \phantom{aaaaaa} \vdots\\
            a_{m1}x_1 + a_{m2}x_2 + \dots + a_{mn}x_n = 0\\
        \end{aligned}
    \end{matrix}
\end{equation*}

Con respecto a las soluciones de los sistemas lineales \textbf{no homogéneos} existen tres posibilidades
\begin{itemize}
    \item[-] Que no tenga soluciones
    \item[-] Que tenga una solución
    \item[-] Que tenga un número infinito de soluciones
\end{itemize}  

Para el sistema general homogéneo la situación es más sencilla. Para sistemas generales no homogéneos, $\mathbf{x_1 = x_2 = \dotsb = x_n = 0}$ es siempre una solución (\textbf{llamada solución trivial o solución cero}), por lo que sólo se tienen dos posibilidades
\begin{itemize}
    \item[-] La solución trivial es la única solución
    \item[-] Existe un número infinito de soluciones además de la trivial.
\end{itemize} 
Las soluciones distintas a la solución cero se llaman soluciones no triviales

\begin{large}
    \textbf{Ejemplo}
\end{large}

\begin{equation*}
    \begin{matrix}
        \begin{aligned}
            x_1 & +2x_2 & +3x_3 & =0\\
            4x_1 & +5x_2 & +6x_3 & =0\\
            3x_1 & +x_2 & -2x_3 & =0
        \end{aligned}
    \end{matrix}
\end{equation*}

\begin{equation}
    \left(\begin{array}{rrr|r}
    1 & 2 & 3 & 0 \\
    4 & 5 & 6 & 0 \\
    3 & 1 & -2 & 0
    \end{array}\right) 
    \xrightarrow{\overset{\begin{aligned} R_2 \rightarrow R_2 - 4R_1 \\ R_3 \rightarrow R_3 - 3R_1\end{aligned}}{}} 
    \left(\begin{array}{rrr|r}
    1 & 2 & 3 & 0 \\
    0 & -3 & -6 & 0 \\
    0 & -5 & -11 & 0
    \end{array}\right)
    \xrightarrow{\overset{\begin{aligned} R_2 \rightarrow \frac{1}{3} R_2 \end{aligned}}{}} 
    \left(\begin{array}{rrr|r}
    1 & 2 & 3 & 0 \\
    0 & 1 & 2 & 0 \\
    0 & -5 & -11 & 0
    \end{array}\right)
\end{equation}

\begin{equation}
    \xrightarrow{\overset{\begin{aligned} R_1 \rightarrow R_1 -2R_2\\ R_3 \rightarrow R_3 + 5R_2\end{aligned}}{}} 
    \left(\begin{array}{rrr|r}
    1 & 0 & -1 & 0 \\
    0 & 1 & 2 & 0 \\
    0 & 0 & -1 & 0
    \end{array}\right) 
    \xrightarrow{\overset{\begin{aligned} R_3 \rightarrow - R_3 \end{aligned}}{}} 
    \left(\begin{array}{rrr|r}
    1 & 0 & -1 & 0 \\
    0 & 1 & 2 & 0 \\
    0 & 0 & 1 & 0
    \end{array}\right)
    \xrightarrow{\overset{\begin{aligned} R_1 \rightarrow R_1 + R_3\\ R_2 \rightarrow R_2 + -2R_3\end{aligned}}{}} 
    \left(\begin{array}{rrr|r}
    1 & 0 & 0 & 0 \\
    0 & 1 & 0 & 0 \\
    0 & 0 & 1 & 0
    \end{array}\right)
\end{equation}

Así, el sistema tiene una solución única (0, 0, 0). Esto es, la única solución al sistema es la trivial.

\begin{large}
    \textbf{Ejemplo 2}
\end{large}

\begin{equation*}
    \begin{matrix}
        \begin{aligned}
            x_1 & +2x_2 & -1x_3 & =0\\
            3x_1 & -3x_2 & +2x_3 & =0\\
            -x_1 & -11x_2 & +6x_3 & =0
        \end{aligned}
    \end{matrix}
\end{equation*}

\begin{equation}
    \left(\begin{array}{rrr|r}
        1 & 2 & -1 & 0 \\
        3 & -3 & 2 & 0 \\
        -1 & -11 & 6 & 0
    \end{array}\right) 
    \xrightarrow{\overset{\begin{aligned} R_2 \rightarrow R_2 - 3R_1 \\ R_3 \rightarrow R_3 + R_1\end{aligned}}{}} 
    \left(\begin{array}{rrr|r}
        1 & 2 & -1 & 0 \\
        0 & -9 & 5 & 0 \\
        0 & -9 & 5 & 0
    \end{array}\right)
    \xrightarrow{\overset{\begin{aligned} R_2 \rightarrow R_2 -R_3\end{aligned}}{}} 
    \left(\begin{array}{rrr|r}
        1 & 2 & -1 & 0 \\
        0 & -9 & 5 & 0 \\
        0 & 0 & 0 & 0
    \end{array}\right)
\end{equation}

Así bien con sustitución hacia atrás podemos darnos cuenta que el conjunto de soluciones es infinito el cual es $(\frac{1}{9} x_3, \frac{5}{9} x_3, x_3)$. Si $x_3 = 0$ se obtiene la \textbf{solución trivial}.

\begin{large}
    \textbf{Sistema Homogéneo con más incógnitas que ecuaciones}
\end{large}


\begin{equation*}
    \begin{matrix}
        \begin{aligned}
            x_1 & +x_2 & -x_3 & =0\\
            4x_1 & -2x_2 & +7x_3 & =0\\
        \end{aligned}
    \end{matrix}
\end{equation*}

\begin{equation}
    \left(\begin{array}{rrr|r}
        1 & 1 & -1 & 0 \\
        4 & -2 & 7 & 0 
    \end{array}\right) 
    \xrightarrow{\overset{\begin{aligned} R_2 \rightarrow R_2 - 4R_1 \end{aligned}}{}} 
    \left(\begin{array}{rrr|r}
        1 & 1 & -1 & 0 \\
        0 & -6 & 11 & 0
    \end{array}\right)
\end{equation}

\begin{equation}
    \xrightarrow{\overset{\begin{aligned} R_2 \rightarrow -\frac{1}{6}R_2 \end{aligned}}{}} 
    \left(\begin{array}{rrr|r}
        1 & 1 & -1 & 0 \\
        0 & 1 & -\frac{11}{6} & 0 
    \end{array}\right) 
    \xrightarrow{\overset{\begin{aligned} R_1 \rightarrow R_1 - R_2 \end{aligned}}{}} 
    \left(\begin{array}{rrr|r}
        1 & 0 & \frac{5}{6} & 0 \\
        0 & 1 & -\frac{11}{6} & 0
    \end{array}\right)
\end{equation}

En esta ocasión \textbf{tenemos más incógnitas que ecuaciones, por lo que hay un número infinito de soluciones}. Si elegimos a $x_3$ como parámetro, encontramos que toda solución es de la forma $(-\frac{5}{6}x_3,\frac{11}{6}x_3, x_3)$

\begin{tcolorbox}[colback=green!20!white,colframe=green!80!black,title=Ecuaciones con más Incógnitas que Ecuaciones]
    En un sistema no homogéneo con más incógnitas que ecuaciones se podía tener o ninguna solución o infinitas soluciones, sin embargo en un sistema homogéneo con más incógnitas que ecuaciones se tienen infnitas solcuiones siempre, más formalmente se tiene:

    \textbf{Teorema -> Condición para tener un número infinito de soluciones}
    El sistema homogéneo tiene un número infinito de solcuiones si $n > m$ 
\end{tcolorbox}
 
\begin{tcolorbox}[colback=red!10!white, colframe=red!70!black, title=Resumen]
    \begin{itemize}
        \item[-] Us sitema Homogéneo de $m$ ecuaciones con $n$ incógnitas es un sistema lineal de la forma
        \item[] \begin{equation*}
            \begin{matrix}
                \begin{aligned}
                    a_{11}x_1 + a_{12}x_2 + \dots + a_{1n}x_n = 0\\
                    a_{21}x_1 + a_{22}x_2 + \dots + a_{2n}x_n = 0\\
                    \vdots \phantom{aaaaaaaa} \vdots \phantom{aaaaaaaaaa} \vdots \phantom{aaaaaa} \vdots\\
                    a_{m1}x_1 + a_{m2}x_2 + \dots + a_{mn}x_n = 0\\
                \end{aligned}
            \end{matrix}
        \end{equation*}

        \item[-] Un sistema lineal homogéneo siempre tiene la \textbf{solución trivial o solución cero} $$x_1 = x_2 = \dotsb = x_n = 0$$
        \item[-] Las soluciones para un sistema lineal homgéneo diferentes de la trivial se denominan \textbf{soluciones no triviales}
        \item[-] El sistema lineal homgéneo anterior tiene un número infinito de soluciones si tiene más incógnitas que ecuaciones ($n > m$)  
    \end{itemize}
\end{tcolorbox}

\end{document}