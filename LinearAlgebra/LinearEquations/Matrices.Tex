\documentclass{article}
\usepackage[top=1cm, left=1.5cm, right=1.5cm, bottom=1.5cm]{geometry}
\usepackage{graphicx}
\usepackage{wrapfig}
\graphicspath{{./img/}}
\usepackage{amsmath}
\usepackage{tcolorbox}
\setlength{\parindent}{0pt}
 
\title{Matrices}
\author{Pablo DS}
\date{06/12/2023}
 
\begin{document}
\maketitle
\section{Matrices}

Las matrices nos permiten escribir sistemas lineales de una manera compacta que facilite la automatización del método de eliminación en una computadora, dándonos un procedimiento rápido y eficaz para determinar las soluciones. Su uso, sin embargo, no nos proporciona solamente la oportunidad de contar con una notación conveniente, sino también de resolver sistemas de ecuaciones lineales y otros problemas computacionales de manera rápida y eficiente, desarrollando operaciones sobre las matrices y trabajando con ellas de acuerdo con las reglas que cumplen.

\begin{tcolorbox}[colback=blue!10!white,colframe=blue!60!black,title=Definición]
    Una matriz A de $m$ x $n$ es un arreglo rectangular de $mn$ números reales (o complejos) ordenados en m filas horizontales y n columnas verticales
\end{tcolorbox}

\begin{alignat*}{2}
    \begin{split}
        A = 
    \end{split} 
& \hspace{ 4em}% 
    \begin{bmatrix}
        a_{11} & a_{12} & \dotsb & \dotsb & a_{1j} & \dotsb & a_{1n}\\
        a_{21} & a_{22} & \dotsb & \dotsb & a_{2j} & \dotsb & a_{2n}\\
        \vdots & \vdots & \dotsb & \dotsb & \vdots & \dotsb & \vdots\\
        a_{i1} & a_{i2} & \dotsb & \dotsb & a_{ij} & \dotsb & a_{in}\\
        \vdots & \vdots &  &  & \vdots &  & \vdots\\
        a_{m1} & a_{m2} & \dotsb & \dotsb & a_{mj} & \dotsb & a_{mn}
    \end{bmatrix} 
\end{alignat*}

El primer subíndice, $i$, indica que estamos trabajando con la i-ésima ecuación, mientras que el segundo subíndice, $j$, está asociado con la j-ésima variable $x_j$.

La \textbf{i-ésima fila} de A es 

\begin{alignat*}{2}
    \begin{bmatrix}
        a_{i1} & a_{i2} & \dotsb & a_{in} 
    \end{bmatrix} 
& \hspace{ 4em}% 
    \begin{split}
        \text{(1 $\leq$ i $\leq$ m)}
    \end{split} 
\end{alignat*}

La \textbf{j-ésima columna} de A es 

\begin{alignat*}{2}
    \begin{bmatrix}
        a_{1j}\\
        a_{2j}\\
        \vdots\\ 
        a_{mj} 
    \end{bmatrix} 
& \hspace{ 4em}% 
    \begin{split}
        \text{(1 $\leq$ j $\leq$ n)}
    \end{split} 
\end{alignat*}

Diremos que A es $m$ por $n$ (que se escribe $m \times n$). Si $m = n$, decimos que A es una \textbf{matriz cuadrada} de orden $n$, y que los números $ a_{11}, a_{22},\dots, a_{mn} $ forman la \textbf{diagonal principal} de A. Nos referimos al número $a_{ij}$, que está en la i-ésima fila y la j-ésima columna de A, como el i, j-ésimo elemento de A, o la entrada (i, j) de A, y lo solemos escribir como $$A = [a_{ij}]$$

Las matrices de 1 $\times n$ o $n \times$ 1 también se denominan como vectores, y lo denotaremos mediante letras minúsculas.

\begin{equation*}
    \mathbf{u}=\left[
    \begin{array}{llll}
        1 & 2 & -1 & 0
        \end{array}\right] \quad \mathbf{v}=\left[\begin{array}{r}
        1 \\
        -1 \\
        3
    \end{array}\right]
\end{equation*}

Si todas las entradas de un vector son iguales a cero, se denota con 0. Además que si A es una matriz de $n \times n$, los filas de A son matrices de 1 $\times n$.

\begin{large}
    Ejemplo 1
\end{large}

La siguiente tabla se puede representar en forma de matriz, la cual proporciona las distancias entre las ciudades indicadas

$$\begin{array}{|c|c|c|c|c|}
    \hline & \text { Londres } & \text { Madrid } & \text { Nueva York } & \text { Tokio } \\
    \hline \text { Londres } & 0 & 785 & 3,469 & 5,959 \\
    \hline \text { Madrid } & 785 & 0 & 3,593 & 6,706 \\
    \hline \text { Nueva York } & 3,469 & 3,593 & 0 & 6,757 \\
    \hline \text { Tokio } & 5,959 & 6,706 & 6,757 & 0 \\
    \hline
\end{array}$$

\begin{large}
    Ejemplo 2
\end{large}

La tabla siguiente, en donde se lista el factor de congelación del viento, muestra cómo una combinación de la temperatura y la velocidad del viento hace que un cuerpo se sienta más frío que la temperatura real. Por ejemplo, cuando la temperatura es de 10 °F y el viento es de 15 millas por hora, el cuerpo pierde la misma cantidad de calor que la que perdería si la temperatura fuera de -18 °F sin viento

$$\begin{tabular}{|r|rrrrrr|}
    \hline & \multicolumn{6}{|c|}{${ }^{\circ} \mathrm{F}$} \\
    $\mathrm{mph}$ & 15 & 10 & 5 & 0 & -5 & -10 \\
    \hline \hline 5 & 12 & 7 & 0 & -5 & -10 & -15 \\
    \hline 10 & -3 & -9 & -15 & -22 & -27 & -34 \\
    \hline 15 & -11 & -18 & -25 & -31 & -38 & -45 \\
    \hline 20 & -17 & -24 & -31 & -39 & -46 & -53 \\
    \hline
\end{tabular}$$

Esta tabla puede representarse como la matriz

$$A=\left[\begin{array}{rrrrrrr}
5 & 12 & 7 & 0 & -5 & -10 & -15 \\
10 & -3 & -9 & -15 & -22 & -27 & -34 \\
15 & -11 & -18 & -25 & -31 & -38 & -45 \\
20 & -17 & -24 & -31 & -39 & -46 & -53
\end{array}\right]$$

\begin{tcolorbox}[colback=blue!10!white,colframe=blue!60!black,title=Matriz Diagonal]
    Es aquella matriz cuadrada A = [$a_{ij}$], en donde cada término fuera de la diagonal principal es igual a cero, es decir, $a_{ij}$ = 0 para $i \neq j$
\end{tcolorbox}

\begin{large}
    Ejemplos
\end{large}

\begin{equation*}
    G=\left[\begin{array}{rr}
    4 & 0 \\
    0 & -2
    \end{array}\right] \quad \text { y } \quad H=\left[\begin{array}{rrr}
    -3 & 0 & 0 \\
    0 & -2 & 0 \\
    0 & 0 & 4
    \end{array}\right]
\end{equation*}

\begin{tcolorbox}[colback=blue!10!white,colframe=blue!60!black,title=Matriz Escalar]
    Es aquella matriz \textbf{diagonal} A = $[a_{ij}]$, en donde todos los términos de la diagonal principal son iguales; es decir, $a_{ij} = c$ para $i = j$ y $a_{ij} = 0$ para $i \neq j$
\end{tcolorbox}

\begin{large}
    Ejemplos
\end{large}

\begin{equation*}
    I_3=\left[\begin{array}{lll}
    1 & 0 & 0 \\
    0 & 1 & 0 \\
    0 & 0 & 1
    \end{array}\right], \quad J=\left[\begin{array}{rr}
    -2 & 0 \\
    0 & -2
    \end{array}\right]
\end{equation*}

\subsection{Operaciones Matriciales}

\begin{tcolorbox}[colback=blue!10!white,colframe=blue!60!black,title=Suma de Matrices]
    Si $A = [a_{ij}]$ y $B = [b_{ij}]$ son matrices de $m \times n$, la suma de A y B da por resultado la matriz $C = [c_{ij}]$ de $m \times n$, definida por $$c_{ij} = a_{ij} + b_{ij} \quad (i \leq i \leq m, 1 \leq j \ n)$$
    Es decir, C se obtiene sumando los elementos correspondientes de A y B
\end{tcolorbox}

Sean
$$A=\left[\begin{array}{lll}
    1 & -2 & 4 \\
    2 & -1 & 3
    \end{array}\right] \quad \text { y } \quad B=\left[\begin{array}{lrr}
    0 & 2 & -4 \\
    1 & 3 & 1
\end{array}\right]$$

Entonces
$$A+B=\left[\begin{array}{lll}
    1+0 & -2+2 & 4+(-4) \\
    2+1 & -1+3 & 3+1
    \end{array}\right]=\left[\begin{array}{lll}
    1 & 0 & 0 \\
    3 & 2 & 4
    \end{array}\right]$$

La suma de las matrices A y B sólo se define cuando A y B tienen el mismo número de filas y el mismo número de columnas; es decir, \textbf{sólo cuando A y B son del mismo tamaño}. Por lo tanto al escribir A + B entendemos que A y B tienen el mismo tamaño. Además la suma de matrices satisface la \textbf{propiedad asociativa. A + (B + C) = (A + B) + C}

\begin{tcolorbox}[colback=blue!10!white,colframe=blue!60!black,title=Multiplicación por un Escalar]
    Si $A = [a_{ij}]$ es una matriz de $m \times n$ y $r$ es un número real, el múltiplo escalar de A por $r$, $rA$, es la matriz $B = [b_{ij}]$ de $m \times n$, donde $$b_{ij} = ra_{ij} (i \leq i \leq m, 1 \leq j \leq n)$$
    Es decir, B se obtiene multiplicando cada elemento de A por $r$
\end{tcolorbox}

Con la suma y Multiplicación de matrices podemos obtener una "resta" de matrices. Si $A$ y $B$ son matrices de $m \times n$, escribimos $A + (-1)B$, lo cual es $A - B$, y denominamos a esto como diferencia de $A$ y $B$.

Sean
$$A=\left[\begin{array}{rrr}
2 & 3 & -5 \\
4 & 2 & 1
\end{array}\right] \quad \text { y } \quad B=\left[\begin{array}{rrr}
2 & -1 & 3 \\
3 & 5 & -2
\end{array}\right]$$

Entonces

$$A-B=\left[\begin{array}{llr}
2-2 & 3+1 & -5-3 \\
4-3 & 2-5 & 1+2
\end{array}\right]=\left[\begin{array}{rrr}
0 & 4 & -8 \\
1 & -3 & 3
\end{array}\right]$$

\begin{large}
    Ejemplo
\end{large}

Sea $\mathbf{p}=\left[\begin{array}{lll}18.95 & 14.75 & 8.60\end{array}\right]$ un vector que representa los precios actuales de tres artículos almacenados en una bodega. Suponga que el almacén anuncia una venta en donde cada uno de estos artículos tiene un descuento de 20 por ciento.
(a) Determine un vector que proporcione el cambio en el precio de cada uno de los tres artículos.
(b) Determine un vector que proporcione los precios nuevos de los artículos.

(a) Como el precio de cada artículo se reduce $20 \%$, el vector proporciona la reducción de los precios para los tres artículos.

$$\begin{aligned}
    0.20 \mathbf{p} & =\left[\begin{array}{lll}
    (0.20) 18.95 & (0.20) 14.75 & (0.20) 8.60
    \end{array}\right] \\
    & =\left[\begin{array}{lll}
    3.79 & 2.95 & 1.72
    \end{array}\right]
\end{aligned}$$


(b) Los precios nuevos de los artículos están dados mediante la expresión

$$ \begin{aligned}
\mathbf{p}-0.20 \mathbf{p} & =\left[\begin{array}{lll}
18.95 & 14.75 & 8.60
\end{array}\right]-\left[\begin{array}{lll}
3.79 & 2.95 & 1.72
\end{array}\right] \\
& =\left[\begin{array}{lll}
15.16 & 11.80 & 6.88
\end{array}\right] 
\end{aligned} $$

Observe que esta expresión también puede escribirse como
$$\mathbf{p}-0.20 \mathbf{p}=0.80 \mathbf{p}$$

\begin{tcolorbox}[colback=blue!10!white,colframe=blue!60!black,title=Combinación Lineal]
    Si $A_1,A_2,\dots, A_k$ son matrices de $m \times n$ y $c_1,c_2,\dots, c_k$son números reales, entonces una expresión de la forma $$c_1A_1 + c_2A_2 + \dots + c_kA_k$$ se denomina \textbf{combinación lineal} de $A_1, A_2,\dots , A_k$. Y $c_1, c_2, \dots, c_k$ se llaman coeficientes.
\end{tcolorbox}

Si

\begin{equation*}
   A_1=\begin{bmatrix}
        \begin{array}{rrr}
            0 &-3 & 5\\
            2 & 2 & 4\\
            1 &-2 &-3\\
        \end{array}
    \end{bmatrix}
    \quad y \quad 
    A_2=\begin{bmatrix}
        \begin{array}{rrr}
            5 & 2 & 3\\
            6 & 2 & 3\\
            -1& -2& 3  
        \end{array}
    \end{bmatrix}
\end{equation*}

entonces $C= 3A_1 - \frac{1}{2}A_2$ es una combinación lineal de $A_1$ y $A_2$

\begin{equation*}
    C= 3 \begin{bmatrix}
        \begin{array}{rrr}
            0 &-3 & 5\\
            2 & 2 & 4\\
            1 &-2 &-3\\
        \end{array}
    \end{bmatrix}
    -\frac{1}{2}\begin{bmatrix}
        \begin{array}{rrr}
            5 & 2 & 3\\
            6 & 2 & 3\\
            -1& -2& 3  
        \end{array}
    \end{bmatrix}
    = \begin{bmatrix}
        \begin{array}{rrr}
            -\frac{5}{2} & -10  & -\frac{27}{2} \\
            3 & 8 & \frac{21}{2} \\
            \frac{7}{2} & -5 & -\frac{21}{2}  
        \end{array}
    \end{bmatrix}
\end{equation*}

\begin{tcolorbox}[colback=blue!10!white,colframe=blue!60!black,title=Transpuesta de una Matriz]
    Si $A = [a_{ij}]$ es una matriz de $m \times n$, la matriz $A^{T} = \left[a_{ij}^{T}\right]$ de $m \times n$, donde $$a_{ij}^{T} = a_{ji} \quad\quad (1\leq i \leq n, 1 \leq j \leq m)$$ es la \textbf{transpuesta} de A. En consecuencia, las entradas en cada fila de $A^T$ son las entradas correspondientes en la column
\end{tcolorbox}

Sean

\begin{equation*}
    A = \begin{bmatrix}
        \begin{array}{rrr}
            4 & -2 & 3\\ 
            0 & 5 & -2
        \end{array}
    \end{bmatrix}
    , \quad B = \begin{bmatrix}
        \begin{array}{rrr}
            6 & 2 & -4\\ 
            3 & -1 & 2\\
            0 & 4 & 3
        \end{array}
    \end{bmatrix}
    , \quad C = \begin{bmatrix}
        \begin{array}{rr}
            5 & 4\\ 
            -3 & 2\\
            2 & -3
        \end{array}
    \end{bmatrix}
    , \quad D = \begin{bmatrix}
        \begin{array}{rrr}
            3 & -5 & 1
        \end{array}
    \end{bmatrix}
    , \quad E = \begin{bmatrix}
        \begin{array}{r}
            2 \\ 
            -1\\
            3
        \end{array}
    \end{bmatrix}
\end{equation*}

Entonces 

\begin{equation*}
    A^T = \begin{bmatrix}
        \begin{array}{rrr}
            4 & 0 \\ 
            -2 & 5 \\
            3 & -2
        \end{array}
    \end{bmatrix}
    , \quad B^T = \begin{bmatrix}
        \begin{array}{rrr}
            6 & 3 & 0\\ 
            2 & -1 & 4\\
            -4 & 2 & 3
        \end{array}
    \end{bmatrix}
    , \quad C^T = \begin{bmatrix}
        \begin{array}{rrr}
            5 & -3 & 2\\ 
            4 & 2 & -3\\
        \end{array}
    \end{bmatrix}
    , \quad D^T = \begin{bmatrix}
        \begin{array}{r}
            3 \\
            -5\\ 
            1
        \end{array}
    \end{bmatrix}
    , \quad E^T = \begin{bmatrix}
        \begin{array}{rrr}
            2 & -1 & 3
        \end{array}
    \end{bmatrix}
\end{equation*}

\end{document}